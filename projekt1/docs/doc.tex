\documentclass[a4paper, 12pt]{article}
\usepackage[left=1.5cm, text={18cm, 25cm}, top=2.5cm]{geometry}
\usepackage[utf8]{inputenc}
\usepackage[czech]{babel}
\usepackage{cite}
\usepackage{graphicx}
\usepackage{float}
\usepackage{amsmath}
\usepackage{tikz}
\usepackage{url}
\usepackage{comment}
\newcommand{\myuv}[1]{\quotedblbase #1\textquotedblleft}
\newcommand{\defVal}[1]{$Default=#1$}

\title{Implementace algoritmu Enumeration sort}
\author{Martin Hruška\\xhrusk16@stud.fit.vutbr.cz}

\date{}
\begin{document}

\maketitle

\section{Úvod}
\label{sec:intro}
Enumeration sort je paralelní řadící algoritmus, který může pracovat s~procesory uspořádanými v~mřížce nebo v~lineárním poli.
Tato práce se zabývá implementací varianty pro lineární pole procesorů, která řadí
posloupnost obsahující $n$ přirozených čísel z~intervalu $<0,255>$ dle jejich velikosti za použití $n+1$ procesorů. 
V~tomto dokumentu bude algoritmus napřed
stručně popsán a bude provedene teoretická analýza jeho složitosti \ref{sec:analysis},
následně budou popsány implementační detaily \ref{sec:impl},
provedené experimenty \ref{sec:exprmts}
a na závěr je uveden sekvenční diagrama znázorňující komunikaci jednotlivých procesorů \ref{sec:seq}.
\cite{forester:home}

\section{Popis a analýza algoritmu}
\label{sec:analysis}
Algoritmus zde bude popsán stručně a neformálně, podrobnější popis lze najít v \cite{prl:pred}.

\section{Implementace}
\label{sec:impl}


\section{Experimenty}
\label{sec:exprmts}

\section{Sekvenční diagram}
\label{sec:seq}

\newpage
\bibliography{literatura}
\bibliographystyle{plain}
\end{document}
